\documentclass[12pt]{article}
\usepackage{amsmath}
\usepackage{amsfonts}
\usepackage[french]{babel}
\usepackage[utf8]{inputenc}
\usepackage[T1]{fontenc}
\usepackage[left=2cm,right=2cm,top=2.5cm,bottom=2.5cm]{geometry}
\usepackage{float}
\usepackage{verbatim}
\usepackage{graphicx}
\usepackage[affil-it]{authblk}
\usepackage{multicol}
\usepackage{hyperref}

\usepackage{listings}
 
\lstset{language=c++}

\author{Scander Mustapha}
\title{Distributed Join Processing\\ on Social Network Data}
\affil{\'Ecole polytechnique}



\begin{document}
\maketitle
\let\olditem\item
\renewcommand{\item}{\olditem[$\bullet$]}

Ce rapport a pour objet de présenter l'implémentation et les résultats du projet "Distributed Join Processing on Social Network Data". Le code est disponible sur le repo \footnote{\url{https://github.com/intermet/inf442}}. Nous avons implémenté une structure de donnée représentant une relation, la jointure séquentielle et distribuée d'atomes , ainsi que la jointure hypercube.

\section{Généralités}
Soient $R_1$ et $R_2$ deux relations sur des entiers naturels.
Notons $a_1 \in \mathbb{N}$ (resp. $a_2 \in \mathbb{N}$) l'arité de $R_1$ (resp. $R_2$).

Considérons des variables $(x_1, \dots, x_{a_1})$ et $(y_1, \dots, y_{a_2})$, éventuellement non distinctes.

\section{Présentation de résultats}

\subsection{Jointure séquentielle}

\subsection{Jointure distribuée Task 5}

\subsection{Jointure distribuée Task 7}

\subsection{Join H}

\end{document}
